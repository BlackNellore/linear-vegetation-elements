\section{Introduction}
The European landscape has dramatically changed during the Holocene as a result of human impact and climatic change \citep{turner1989landscape, marquer2017quantifying}. Especially since the industrial revolution, landscapes have been deforested and reshaped into rural and agricultural landscapes. These are dominated by a mosaic of grasslands, forests and urban areas, separated or connected by linear landscape elements such as roads, ditches, tree lines, lynchets and hedgerows \citep{bailly2008agrarian, meyer2012multifunctional, van2013modelling}. The distribution, abundance and richness of species in these landscapes is related to the amount, height, length and quality of linear vegetation elements \citep{aguirre2016functional, spellerberg1999introduction, croxton2005linear}. The same holds true for the dispersal of seeds and the flow of matter, nutrients and water \citep{turner1989landscape, burel1996hedgerows}. Additionally, linear infrastructures such as roads and railways form barriers which lead to habitat fragmentation. In contrast, green lanes which are flanked by hedges and/or tree lines may form connecting corridors. Hence, linear vegetation elements are of key importance for biodiversity in agricultural landscapes. A wider audience has become aware that historic agricultural practices are part of the cultural heritage \citep{jongman2004landscape, gobster2007shared} and need to be conserved. However, the occurrence of  green lanes and hedgerows has strongly diminished in many countries \citep{boutin2001hedgerows, stoate2001ecological}. This is mostly a consequence of larger agricultural fields, monocultures and a reduction in non-crop features which reduces the complexity and diversity of landscape structure \citep{croxton2005linear}. Detailed knowledge of the spatial occurrence, current status, frequency and ecological functions of linear vegetation elements in a landscape is therefore of key importance for biodiversity conservation and regional planning.

The mapping of linear vegetation elements has traditionally been done with visual interpretations of aerial photographs in combination with intensive field campaigns \citep{aksoy2010automatic}. However, this approach is time-consuming and has limited transferability to larger areas. New methods have therefore been developed that use raster images to map linear vegetation elements by using their spectral properties in visible or infrared wavelengths, e.g. from SPOT, ASTER and Landsat imagery \citep{thornton2006sub, vannier2014multiscale,tansey2009object}. This allows an automated and hierarchical feature extraction from very high resolution imagery \citep{aksoy2010automatic}. Despite these developments, comprehensive high-resolution inventories of linear vegetation elements such as hedgerows and tree lines are lacking at regional and continental scales. The lack of such high-resolution measurements of 3D ecosystem structure across broad spatial extents impedes major advancements in animal ecology and biodiversity science, e.g. for predicting animal species distributions \citep{kissling2017eecolidar}. On a European scale, density maps of linear vegetation elements (and ditches) have been produced at 1 km$^2$ resolution through spatial modeling of 200,000 ground observations \citep{van2013modelling}. However, these maps strongly depend on spatial interpolation methods as well as regional environmental and socio-economic variation and therefore contain a considerable amount of uncertainty in the exact spatial distribution of linear vegetation elements in the landscape. High-resolution measurements of 2D and 3D ecosystem structures derived from cross-national remote sensing datasets are therefore needed to identify and map linear vegetation elements across broad spatial extents \citep{kissling2017eecolidar}.

An exciting development for quantifying 3D ecosystem structures is the increasing availability of high-resolution remote sensing data derived from Light Detection and Ranging (LiDAR) \citep{lim2003lidar}. LiDAR data have important properties which are useful for the detection, delineation and 3D characterization of vegetation, such as their physical dimensions x, y, z, laser return intensity, and multiple return information \citep{lefsky2002lidar, eitel2016beyond}. Vegetation partly reflects the LiDAR signal and usually generates multiple returns, including a first return at the top of the canopy and a last return on the underlying terrain surface. This provides valuable information for separating vegetation from non-vegetation \citep{lim2003lidar}. Moreover, the intensity values describe the strength of the returning light, which depends on the type of surface on which it is reflected and therefore provides information on the surface composition \citep{song2002assessing}. The shape and internal structure of vegetation can be analyzed by classifying information from the different return values and a variety of features, which can be calculated from the point cloud \citep{lim2003lidar, weinmann2015semantic}. Some applications of using airborne LiDAR data to quantify linear elements in agricultural landscapes already exist, e.g. the extraction of ditches in a Mediterranean vineyard landscape \citep{bailly2008agrarian}. However, the characterization of linear vegetation elements in rural and agricultural landscapes from LiDAR point clouds is mostly lacking. Nevertheless, the increasing availability of nation-wide and freely accessible LiDAR data in several European countries provides exciting new avenues for characterizing 3D vegetation structures in agricultural landscapes \citep{kissling2017eecolidar}.

Here, we present a transparent and accurate method for classifying linear vegetation elements from LiDAR point clouds in an agricultural landscape. We develop the method using free and open source data and analysis tools and apply it for characterizing various linear vegetation elements in a rural landscape of the Netherlands containing agricultural fields, grasslands, bare soil, roads and buildings. While the identification of linear objects (e.g. the automated delineation of roads) is often based on raster-based remotely sensed imagery \citep{quackenbush2004review}, the detection of linear vegetation objects is more complex due to their 3-dimensional shape, size and variety. We therefore use a method which allows us to directly classify the point cloud using machine learning algorithms \citep{yan2015urban}. Using fourteen features based on echo, local geometric and local eigenvalue information of the LiDAR point cloud, we apply a machine learning algorithm to classify the vegetation points in the point cloud. We then use a region growing algorithm to segment the classified vegetation points into rectangular objects, and apply elongatedness as a criterion to classify linear objects. The accuracy of the method is tested against manually annotated datasets based on high resolution orthophotos and field surveys. Our method provides a promising first step for upscaling the detection of linear vegetation objects in agricultural landscapes to broad spatial extents.